
\documentclass[conf]{IEEEtran}

\title{IPDPS 2026 Tutorial Proposal}

\usepackage{ipdps_tutorial}

\usetag{explanation}

\begin{document}
\maketitle

\tutexpl{IMPORTANT: Please do not modify the format of this template.
Comment out the usetag line above to remove the explanations.}

\tutsection{Tutorial Facts}

\tutpart{Title}{How To Write a Tutorial Proposal for IPDPS}
\tutpart{Duration}{half-day or full-day}
\tutpart{Abstract}{
This can be longer, but not too long.
  
Up to 200 words.
}

\tutpart{Goals}{
  \tutexpl{the main learning objectives and the topics to be covered}
}

\tutpart{Structure}{
  \tutexpl{the format of delivery (e.g., lecture, hands-on exercises)}
}

\tutpart{Schedule}{
  \tutexpl{a section-by-section breakdown with estimated time allocation}
}
  
\tutpart{Pre-requisites}{
  \tutexpl{the prior knowledge expected from attendees and required materials (e.g., laptop)}
}


\tutsection{Presenter Information}

\tutexpl{One presenter (lead) should be designated as the primary contact for correspondence.}

\begin{tutpresenters}
\tutleadpresenter{Firstname Lastname1}{affiliation1}{email1@domain}
\tutpresenter{Firstname Lastname2}{affiliation2}{email2@domain}
\tutpresenter{Firstname Lastname3}{affiliation3}{email3@domain}
\end{tutpresenters}
 

\tutbio{Firstname Lastname1}{  
  \tutexpl{This is a brief biography of the  presenter. 
  It should include relevant experience and expertise related to the tutorial topic. 
  Up to 200 words.}
}

\tutbio{Firstname Lastname2}{
  \tutexpl{This is a brief biography of the  presenter. 
  It should include relevant experience and expertise related to the tutorial topic. 
  Up to 200 words.}
}

\tutbio{Firstname Lastname3}{
  \tutexpl{This is a brief biography of the  presenter. 
  It should include relevant experience and expertise related to the tutorial topic. 
  Up to 200 words.}
}

\end{document}
